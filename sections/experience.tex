\documentclass[letter,10pt]{article}
\usepackage{TLCresume}
\begin{document}


%====================
% EXPERIENCE A
%====================
\subsection{{FLIGHT DYNAMICS SYSTEMS ENGINEER IN THE INTERPLANETARY MISSION ANALYSIS TEAM \hfill Feb. 2025  --- Currently}}
\subtext{Deimos Space - Indra \hfill Tres Cantos, Comunidad de Madrid, Spain (Hybrid)}
\begin{zitemize}
\item Development of Flight Dynamics System C++ astrodynamics library for LEO-PNT satellite. 
\item Maintenance of the continuous integration system (bamboo), and deployment infrastructure (CMake).
\item Enhancement of an Object Relational Mapping (ORM) tool between PostgreSQL and C++ that enables the system's domain logic.
\end{zitemize}

\subsection{{FLIGHT DYNAMICS ENGINEER IN THE SPACE SITUATIONAL AWARENESS (SSA) TEAM \hfill Apr. 2023  --- Feb. 2025}}
\subtext{GMV \hfill Tres Cantos, Comunidad de Madrid, Spain (Hybrid)}
\begin{zitemize}
\item Development of C++ astrodynamics library to build and maintain a space catalogue of objects.
\begin{zitemize}
    \item Low-level implementation of astrodynamic algorithms, and extensive validation through testing.
    \item High-level analysis of the functionality and limitations of the algorithms implemented.
\end{zitemize}
\item Integration of the cataloguing library infrastructure in the final system:
\begin{zitemize}
    \item Support with back-end development. Maintenance and improvement of subsystem tests. Creation of performance (stress, load) tests for key back-end processes. 
    \item Integration of a monitoring stack for external components: database, message orquestration, micro-services. Monitoring of the internal system KPI's using scrape agents and monitoring tools.
\end{zitemize}
\item Some technologies used: C++17 (Boost, Eigen, CMake, Google Tests), Visual Studio Code, GitLab, Python (pandas, plotly), Spring Boot, Kafka, PostgreSQL, RESTful API (OpenAPI), Robot framework, Grafana, and Prometheus.

\item International work environment with members from Germany, France, and Spain. Agile methodology (SCRUM).
\end{zitemize}
%=====================
% EXPERIENCE B
%====================
\subsection{{INTERN IN THE SPACE SITUATIONAL AWARENESS (SSA) TEAM \hfill Jun. 2022  --- Apr. 2023}}
\subtext{GMV \hfill Tres Cantos, Comunidad de Madrid, Spain (Hybrid)}
\begin{zitemize}
\item Performing Master's thesis: \href{https://repository.tudelft.nl/islandora/object/uuid:f472201e-0e32-4b9e-8aa3-04521908396a}{Assimilation of Swarm C atmospheric density observations into NRLMSISE-00}. Analysis of the accuracy improvement of data assimilation into a density model with several satellite geometries at varying altitudes and space weather conditions. Preliminary results presented in \href{https://conference.sdo.esoc.esa.int/proceedings/neosst2/paper/90/NEOSST2-paper90.pdf}{NEO-SST 2 conference}.
\end{zitemize}
%====================
% EXPERIENCE C
%====================
\subsection{{INTERN IN THE ADVANCED CONCEPTS TEAM (ACT) \hfill Jul. 2020  --- Nov. 2020}}
\subtext{European Space Agency (ESA) \hfill Noordwijk, Zuid Holland, The Netherlands (Hybrid)}
\begin{zitemize}
\item Created three optimisation challenges in the web platform \href{https://optimize.esa.int/}{optimize}: Jupiter Icy Moons Explorer (JUICE) mission design, Traveling Salesman Problem (TSP) based on space debris recovery, and interferometry reconstruction.
\end{zitemize}

%====================
% EXPERIENCE E
%====================
%\subsection{{ROLE / PROJECT E \hfill MMM YYYY --- MMM YYYY}}
%\subtext{company E \hfill somewhere, state}
%\begin{zitemize}
%\item In lobortis libero consectetur eros vehicula, vel pellentesque quam fringilla.
%\item Ut malesuada purus at mi placerat dapibus.
%\item Suspendisse finibus massa eu nisi dictum, a imperdiet tellus convallis.
%\item Nam feugiat erat vestibulum lacus feugiat, efficitur gravida nunc imperdiet.
%\item Morbi porta lacus vitae augue luctus, a rhoncus est sagittis.
%\end{zitemize}

\end{document}